\documentclass{article}
\usepackage[utf8]{inputenc}
\usepackage{graphicx}
\usepackage{enumitem}
\usepackage{multicol}
\usepackage{mathtools, amssymb, amsthm} % imports amsmath

\newtheorem{theorem}{Theorem}
\newtheorem{lemma}[theorem]{Lemma}
\newtheorem{proposition}[theorem]{Proposition}
\newtheorem{corollary}[theorem]{Corollary}

\theoremstyle{definition}
\newtheorem{definition}[theorem]{Definition}
\newtheorem{example}[theorem]{Example}

\theoremstyle{remark}
\newtheorem{remark}[theorem]{Remark}
\newenvironment{problem}[2][Problem]{\begin{trivlist}\item[\hskip \labelsep {\bfseries #1}\hskip \labelsep {\bfseries #2.}]}{\end{trivlist}}


\author{Daniel Palma}
\date{\today}
\title{STA4724: Big Data Analytics Methods}

\begin{document}

\maketitle
\newpage

\vspace*{\fill}

\begin{center}
    \textbf{Course Update on February 6th, 2025}
\end{center}

Dr. Tomova spoke with Dr. Ding, and had a discussion about the course. there needed to be a change but they all agreed to NOT change the course in the middle of the semester.

They agreed to figure out how to balance the pre-requisites, based on the name the course should include a lot of concepts of linear algebra, however they dont want it to become a barrier for students currently taking the course (they want students to keep attending) so they will be currently making it similar to previous semesters, consistent with the syllabus and include more practical applications, and less conceptual theory moving forward.

The last three topics, 10-12 (Projection and Isometry, Variance-Covariance Matrix, and Multivariate Normal Distribution) are not going to be covered and will be optional. today we're finishing up 8-9 (spectral decomposition, Properties and Derivations of Matrix Traces). The following homeworks and exams will not have any proof questions

\vspace*{\fill}

\newpage

\tableofcontents
\newpage

% Example proof structure:
% \begin{theorem}
%     State your theorem here.
% \end{theorem}
% \begin{proof}
%     Proof goes here.
% \end{proof}
\section{Matrix Algebra}

\subsection{Definitions of Matrices and Vectors}

\paragraph{Matrix}
\begin{itemize}
    \item a matrix is an arrangement of numbers in rectangular form
    \item a $J \times K$ matrix has $J$ rows and $k$ columns
    \item a Square matrix is of order $(2,2)$ as a special case
    \item Vectors are subcategories of matrices that have either one row or one column
    \subitem $(1,k)$ is one rowm, and multiple columns, e.g. $\begin{bmatrix} 1 & 2 & 3 \end{bmatrix}$
    \subitem $(k,1)$ is one column, and multiple rows, e.g. $\begin{bmatrix} 1 \\ 2 \\ 3 \end{bmatrix}$
    \item a matrix with one row and one column is the same as a scalar. $a = 5 \Leftrightarrow a = \begin{bmatrix}
        5
    \end{bmatrix}$
\end{itemize}

\subsection{Addition, Subtraction, Multiplication}

\begin{itemize}
    \item $A + B = C$
    \item $A + B \Leftrightarrow B + A$
    \item $(A + B) + C \Leftrightarrow A + (B + C)$
\end{itemize}

\paragraph{Transposition}
An order $(j,j)$ matrix is said to be symmetric if $A = A^T$

\begin{itemize}
    \item $(A^T)^T \Leftrightarrow A$
    \item $(kA)^T \Leftrightarrow kA^T$ where $k$ is a scalar
    \item $(A + B)^T \Leftrightarrow A^T + B^T$
    \item $kA \Rightarrow k \cdot \begin{bmatrix}
        a_1 & a_2 & a_3
    \end{bmatrix} \Rightarrow \begin{bmatrix}
        k \cdot a_1 & k \cdot a_2 & k \cdot a_3
    \end{bmatrix}$
    \item Given matrix A of order $(m,n)$ and matrix B of order $(n,r)$
    \subitem $C = A \cdot B$ is of order $(m,r)$ = $\begin{bmatrix}
        C_{mr}
    \end{bmatrix}$ where $C_{mr} = \sum_{i=1}^{n} A_{mi} \cdot B_{ir}$
\end{itemize}

\begin{example}
    Given the matrices $A = \begin{bmatrix}
        1 & 2 & 3 \\
        4 & 5 & 6
    \end{bmatrix}$ and $B = \begin{bmatrix}
        7 & 8 \\
        9 & 10 \\
        11 & 12
    \end{bmatrix}$, find $C = A \cdot B$

    \begin{align*}
        C_{11} &= 1 \cdot 7 + 2 \cdot 9 + 3 \cdot 11 = 58 \\
        C_{12} &= 1 \cdot 8 + 2 \cdot 10 + 3 \cdot 12 = 64 \\
        C_{21} &= 4 \cdot 7 + 5 \cdot 9 + 6 \cdot 11 = 139 \\
        C_{22} &= 4 \cdot 8 + 5 \cdot 10 + 6 \cdot 12 = 154
    \end{align*}

    Therefore, $C = \begin{bmatrix}
        58 & 64 \\
        139 & 154
    \end{bmatrix}$
\end{example}

\paragraph{Properties}
\begin{itemize}
    \item $AB \neq BA$
    \item $A(BC) \Leftrightarrow (AB)C$
    \item $A(B + C) \Leftrightarrow AB + AC$ 
    \item $(AB)^T \Leftrightarrow B^TA^T$
    \item $A^n \Leftrightarrow A_0 \cdot A_1 \cdot ... \cdot A_{n-1}$
\end{itemize}

\subsection{Diagonal and Identity Matrices}

\paragraph{Diagonal matrix}
A diagonal matrix is a square matrix with zero entries except possible on the main diagonal

\begin{example}
    $\begin{bmatrix}
        1 & 0 & 0 \\
        0 & 1 & 0 \\
        0 & 0 & 1
    \end{bmatrix}$ is a diagonal matrix. note that they dont need to be 1s, they can be any number, including zero. 
\end{example}

In general, a diagonal matrix is given by $D_{mn} = \begin{bmatrix}
    d_1 & 0 & 0 & ... & 0 \\
    0 & d_2 & 0 & ... & 0 \\
    0 & 0 & d_3 & ... & 0 \\
    ... & ... & ... & ... & ... \\
    0 & 0 & 0 & ... & d_n
\end{bmatrix}$

\paragraph{Echelon Form}

\begin{enumerate}
    \item row echeleon form (ref)
    \subitem The first non-zero element in each row is called the leading entry, is always 1
    \subitem Each leading entry is in a column to the right of the leading entry in the previous row (if any)
    \subitem Rows with all zero elements are below rows with non-zero elements (if any)
    \item reduced row echelon form (rref)
    \subitem any ref with the leading entry in each row is the only non-zero entry in its column. 
\end{enumerate}

\paragraph{Properties of Diagonal Matrices}

\begin{itemize}
    \item A diagonal matrix $D$ is invertible if and only if all the diagonal elements are non zero.
\end{itemize}

\begin{example}
    given $D^{-1} = \begin{bmatrix}
        1/d_1 & 0 & ... & 0 \\
        0 & 1/d_2 & ... & 0 \\
        ... & ... & ... & ... \\
        0 & 0 & ... & 1/d_n 
    \end{bmatrix}$

    so $DD^{-1} = \begin{bmatrix}
        d_1 \cdot 1/d_1 & 0 & ... & 0 \\
        0 & d_2 \cdot 1/d_2 & ... & 0 \\
        ... & ... & ... & ... \\
        0 & 0 & ... & d_n \cdot 1/d_n
    \end{bmatrix} \rightarrow I $ which is the identity matrix  
\end{example}

\paragraph{Identity Matrix}
The identity matrix is a square matrix, consisting of ones along the diagonal and zeros elsewhere. Typically, $I$ is used to denote the identity matrix. 

\begin{example}
    $$I_{nn} = \begin{bmatrix}
        1 & 0 & 0 & ... & 0 \\
        0 & 1 & 0 & ... & 0 \\
        0 & 0 & 1 & ... & 0 \\
        ... & ... & ... & ... & ... \\
        0 & 0 & 0 & ... & 1
    \end{bmatrix}$$
\end{example}

\paragraph{Properties of Identity Matrices}
\begin{itemize}
    \item $AI = IA = A$
\end{itemize}

\paragraph{Zero Matrix}
a zero matrix consists of all zero elements.

\subsection{Determinant and Eigenstructure}

\paragraph{Determinant}
Determinants are defined only for square matrices and scalars.

\begin{example}

    let $A = \begin{bmatrix}
        a & b \\
        c & d
    \end{bmatrix}$, then 
    det($A$) $ = ad - cb$
\end{example}

the determinant of a matrix is denoted by $|A|$ or det($A$) and is a number which encodes a lot of information about the matrix.

In general, we need to first define the cofactor $Q_{r,s}$ of each element of A, $a_{r,s}$. The cofactor of $a_{r,s}$ is $Q_{r,s} = -1^{r+s} |A_{r,s}|$ (where $|A_{r,s}|$ is the determinant of the matrix obtained by deleting the $r$-th row and $s$-th column of $A$).

The last step is to define the determinant of the matrix A as $$|A| = \sum_{j=1}^{n}a_{ij}Q_{ij}$$ or $$|A| = \sum_{i=1}^na_{ij}Q_{ij}$$

\paragraph{Properties of Determinants}

\begin{itemize}
    \item $|I| = 1$
    \item if exchanging two rows of a matrix, we only need to reverse the sign of its determinant
    \item If we multiply one row of a matrix by a scalar $k$, the determinant is also multiplied by $k$. 
    \item The determinant behaves like a linear function on the rows of the matrix  
\end{itemize}

\begin{lemma}
    The geometric multiplicity of an eigenvalue is at most its algebraic multiplicity. 
\end{lemma}

Characteristic equation $det(A - \lambda I) = 0$

\paragraph{The Geometric Multiplicity of Eigenvalues}
\begin{itemize}
    \item It is the dimension of the linear space of its associated eigenvectors.
\end{itemize}

Let $A$ be a $k \times k$ matrix, $\lambda_k$ be one of the eigenvalues of $A$ and denote its associated eigenspace by $E_k$. Then the dimension of $E_k$ is called the geometric multiplicity of this eigenvalue $\lambda_k$ 


\begin{proof}
    Suppose that the geometric multiplicity of $\lambda_k$ is equal to $g$, so that there are $g$ linearly independent eigenvectors. $x_1,...x_g$ associated to $\lambda_k$. Randomly choose $k-g$ factors $x_{g+1}... x_k$, all having dimension $k \times l$ and such that the $k$ column vectors $x_1, ... ,x_k$ are linearly independent. \\

    Define the $k \times k$ matrix $$x = [x_1, ... , x_k]$$ for any $g$, denoted by $b_g$ the vector that solves $xb_g = Ax_g = \lambda x_g$ \\ 

    Define the $k \times (k-g)$ matrix $$B = [b_g+1, ... , b_k]$$

    and denote by $C$ its upper $g \times (k-g)$ block, and denote by $D$ its lower $(k-g) \times (k-g)$ block $$B = \begin{bmatrix}
        C \\ D
    \end{bmatrix}$$

    Denote by I the $k \times k$ identity matrix. for any scalar $\lambda$, we have that $(A- \lambda I)X$ ($= 0$ to find $x$ for $\lambda_k \times$) \\ 

    
    
\end{proof}

I dont understand anything she wrote after this so sorry that there isnt anything here

\subsection{Inverses and Singularity}

Suppose $A$ is a square matrix. we look for an Inverse Matrix, $A^{-1}$ of the same size, such that $AA^{-1} \Rightarrow 1$ (does nothing to a vector) 

Thus, $AA^{-1}x = x$

Multiplying $Ax = b$ by $A^{-1}$ gives $A^{-1}Ax = A^{-1}b \Rightarrow x = A^{-1}b$

If the determinant of $A$ is non-zero, then $A^{-1}$ exists, thus it is invertible. 

\bigskip
\bigskip
\bigskip

I also dont know what she wrote for this sorry guys 

\subsection{Systems of Equations}

\paragraph{Systems of linear equations.} 

A system of $K$ linear equations in $L$ unknown variables is a set of equations of the form:

\begin{align*}
    a_{11}x_1 + a_{12}x_2 + ... + a_{1L}x_L &= b_1 \\
    a_{21}x_1 + a_{22}x_2 + ... + a_{2L}x_L &= b_2 \\
    ... \\
    a_{K1}x_1 + a_{K2}x_2 + ... + a_{KL}x_L &= b_K
\end{align*}

and can be represented by the matrix equation $Ax = b$ where $A$ is a $K \times L$ matrix, $x$ is a $L \times 1$ vector, and $b$ is a $K \times 1$ vector.

\begin{definition}
    An Augmented Matrix is a matrix obtained by appending the columns of two matrices that have the same number of rows.
\end{definition}

Let $A$ be a $K \times L$ matrix, $B$ is a $K \times M$ matrix. The augmented matrix of $A$ and $B$ is denoted by $[A|B]$ and is a $K \times (L+M)$ matrix. and is obtained by appending the columns of $B$ to the right of $A$. 

\paragraph{Homogeneous System} 

The vector of constants on the right-hand side of the equation is zero. $Ax = 0$

\paragraph{Non-Homogeneous System}

The vector of constants on the right-hand side of the equation is not zero. $Ax = b$

By elementary row operations, a non-Homogeneous system can be transformed into an $Rx = b_R$ where the coefficient matrix $R$ is in row echelon form. If $R$ has a zero row $R_i$ with $b_{Ri} \neq 0$, then the system has no solution. 

\paragraph{Partitioned System}

Suppose that we have a $K \times L$ row ecehlon form matrix $R$ with $r$ basic columns and the last $L - r$ columns are non-basic. Partition the matrix into two blocks $R = [B~~N]$ where $B$ is a $K \times r$ matrix and $N$ is a $K \times (L-r)$ matrix, Similarly partition the vector of variables into two blocks $x = [x_B~~x_N]$ where $x_B$ is a $r \times 1$ vector and $x_N$ is a $(L-r) \times 1$ vector.

\subsection{Singular Value Decomposition (SVD)}

\paragraph{Calculation}
The columns of $u$ are the eigenvectors corresponding to the nonzero values of $MM^T$ 

I did not get the rest of these notes

\begin{proof}
    Since $MM^T = (u\Sigma v^t)\cdot(v \Sigma u^T)$ 
\end{proof}

\subsection{Spectral Decomposition}

\paragraph{Calculation}

Let $A$ be a square $n \times n$ matrix $(\lambda_i, v_i)$.

\begin{align*}
    A = \begin{bmatrix}
        v_1 & v_2 & ... & v_n
    \end{bmatrix} \begin{bmatrix}
        \lambda_1 & 0 & ... & 0 \\
        0 & \lambda_2 & ... & 0 \\
        ... & ... & ... & ... \\
        0 & 0 & ... & \lambda_n
    \end{bmatrix} \begin{bmatrix}
        v_1 \\ v_2 \\ ... \\ v_n \end{bmatrix} = \sum_{i=1}^{n} \lambda_i v_i v_i^T
\end{align*}

\paragraph{Cholesky Decomposition} 

$$A = VA^{\frac{1}{2}}(VA^{\frac{1}{2}})^T, A^{\frac{1}{2}} = diag(\lambda_1^{\frac{1}{2}},...,\lambda_n^{\frac{1}{2}}) = LL^*$$

$$A_{tv} = \sum_{u=1}^k L_{tu}(L^*)_{uv} = \sum_{u=1}^k L_{tu}L_{vu}$$

i didnt get the rest



\subsection{Properties and Derivations of Matrix Traces}

\paragraph{Vector Norms}

For any vector $x \in \mathbb{R}^n$, the $l_2$ norm of $x$ is given by $$||x||_2 = \sqrt{\sum_{i=1}^n x_i^2}$$

\paragraph{Inner Product}

for any $x = \begin{bmatrix}
    x_1 \\ x_2
\end{bmatrix}$ and $y = \begin{bmatrix}
    y_1 \\ y_2
\end{bmatrix}$, the angle $\theta$ can be computed by $$\cos(\theta) = \frac{x_1y_1 + x_2y_2}{\sqrt{x_1^2x_2^2}\sqrt{y_1^2y_2^2}}$$

let $\langle x,y \rangle$ be the inner product of $x$ and $y$, then $$\langle x,y \rangle = x^Ty$$ where $$\langle x,y \rangle = \sum_{i=1}^m\sum_{j=1}^nx_{ij}y_{ij}$$

\paragraph{Matrix Trace}

For any square matrix $x \in \mathbb{R}^{n\times n}$, the matrix trace $tr()$ is the sum of the diagonal entries of $$tr(x) = \sum_{i=1}^n x_{ii}$$ where $x_{ii}, \forall i \in [n]$ is the (i,i)-th entry of $x$. Thus, $tr(x) = tr(x^T)$

\paragraph{Properties of Matrix Trace}

\begin{itemize}
    \item $tr(x+y) = tr(x) + tr(y)$
    \item $tr(\alpha x + \beta y) = \alpha tr(x) + \beta tr(y)$
    \item $tr(xy) = tr(yx)$
    \item $tr(x^Tx) = ||x||^2_F$
\end{itemize}

\subsection{Projection and Isometry}

OPTIONAL - SKIPPED

\subsection{Variance-Covariance Matrix}

OPTIONAL - SKIPPED

\subsection{Multivariate Normal Distribution}

OPTIONAL - SKIPPED

\section{Machine Learning}





\end{document}